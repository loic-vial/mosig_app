\documentclass[11pt]{article}

\usepackage[french]{babel}
\usepackage[utf8]{inputenc}
\usepackage{fancyhdr}
\usepackage{lastpage}
\usepackage{graphicx}
\usepackage{amsmath}

%\usepackage{graphvizzz}

%\usepackage[usenames,dvipsnames]{pstricks}
%\usepackage{epsfig}
%\usepackage{pst-grad} % For gradients
%\usepackage{pst-plot} % For axes

\usepackage{enumitem}

\usepackage{framed}

%%%%%%
% Pour mise-en-forme des fichiers Ada
%
% voir exemple en fin de ce fichier.
%
% ATTENTION, requiert encoding utf-8 (voir 2ième "\lstset" ci-dessous)
 
\usepackage{listings}
\lstset{
  morekeywords={abort,abs,accept,access,all,and,array,at,begin,body,
      case,constant,declare,delay,delta,digits,do,else,elsif,end,entry,
      exception,exit,for,function,generic,goto,if,in,is,limited,loop,
      mod,new,not,null,of,or,others,out,package,pragma,private,
      procedure,raise,range,record,rem,renames,return,reverse,select,
      separate,subtype,task,terminate,then,type,use,when,while,with,
      xor,abstract,aliased,protected,requeue,tagged,until,printf},
  sensitive=f,
  morecomment=[l]--,
  morestring=[d]",
  showstringspaces=false,
  basicstyle=\small\ttfamily,
  keywordstyle=\bf\small,
  commentstyle=\itshape,
  stringstyle=\sf,
  extendedchars=true,
  columns=[c]fixed
}

% CI-DESSOUS: conversion des caractères accentués UTF-8 
% en caractères TeX dans les listings...
\lstset{
  literate=%
  {À}{{\`A}}1 {Â}{{\^A}}1 {Ç}{{\c{C}}}1%
  {à}{{\`a}}1 {â}{{\^a}}1 {ç}{{\c{c}}}1%
  {É}{{\'E}}1 {È}{{\`E}}1 {Ê}{{\^E}}1 {Ë}{{\"E}}1% 
  {é}{{\'e}}1 {è}{{\`e}}1 {ê}{{\^e}}1 {ë}{{\"e}}1%
  {Ï}{{\"I}}1 {Î}{{\^I}}1 {Ô}{{\^O}}1%
  {ï}{{\"i}}1 {î}{{\^i}}1 {ô}{{\^o}}1%
  {Ù}{{\`U}}1 {Û}{{\^U}}1 {Ü}{{\"U}}1%
  {ù}{{\`u}}1 {û}{{\^u}}1 {ü}{{\"u}}1%
}

%%%%%%%%%%
% TAILLE DES PAGES (A4 serré)

\setlength{\parindent}{20pt}
\setlength{\parskip}{1ex}
\setlength{\textwidth}{17cm}
%\setlength{\textwidth}{16cm}
\setlength{\textheight}{23cm}
\setlength{\oddsidemargin}{-.7cm}
\setlength{\evensidemargin}{-.7cm}
\setlength{\topmargin}{-.5in}

%%%%%%%%%%
% EN-TÊTES ET PIED DE PAGES

\pagestyle{fancyplain}
\renewcommand{\headrulewidth}{0pt}
\addtolength{\headheight}{1.6pt}
\addtolength{\headheight}{2.6pt}
\lfoot{}
\cfoot{}


%%%%%%%%%%
% titre du document

\title{Algorithms \\
	\textbf{``Hold'em for n00bs''}}

\author{Thanh Luan, Six Cyril, Vial Loïc \\
			Rouby Thomas, Marriott Richard\\
			Poupin Pierre} 

\date{7\up{th} of November 2014}


\begin{document}

\maketitle

\section{Introduction}
Talk about the game and dynamic programming.

The game we are playing is quite simple : It is a turn based two-player game, using a line of cards : Each player can take either the left-most card or the right-most card.
The winner is the one who gets the highest value calculated by the sum of cards.
We assume that we play with a basic artificial intelligence, which always pick the highest valued card.

We will present several types of algorithms to win in most cases : a greedy one, a full-exploration one, and a dynamic one.

The greedy algorithm is simply what the IA does, we hope to win at the end, but we are not sure if we will succeed, whereas the full-exploration of space's method checks every combination of cards' picking, to find a solution in advance, and then we can apply it.
The dynamic programming algorithm is basically optimizations we made on the full exploration algorithm, to enhance the performances and lessen the costs in memory and time.

We will now present with more details those algorithms.


\section{The game algorithm}
% Describe the game, show an example

\section{The greedy algorithm}
\subsubsection{Explanation}
\subsubsection{Limitations and advantages}

\section{Full solution-space algorithm}
% Cyril's algorithm
\subsubsection{Explanation}
\subsubsection{Complexity, optimization}

\section{Dynamic implementation}



\section{Conclusion}





%	\digraph a {
%		size="3.5,3.5";
%		a->b;
%		a->c;
%		b->d;
%		b->e;
%		c->h;
%		c->i;
%		d->k;
%		e->k;
%		h->l;
%		i->l;
%		l->m;
%		k->m;
%	}
\end{document}
