\documentclass[a4paper,10pt]{report}

%%%% PRATIQUE POUR LES ALINEAS CHIANTS
\usepackage{indentfirst}

\usepackage[colorlinks=false]{hyperref}

%%%% POUR L'OPTION LABEL= %%%
\usepackage{enumitem}

\setlength{\parindent}{30pt}
\setlength{\parskip}{1ex}
\setlength{\textwidth}{15cm}
\setlength{\textheight}{24cm}
\setlength{\oddsidemargin}{0.2cm}
\setlength{\evensidemargin}{-.7cm}
\setlength{\topmargin}{-.5in}

\usepackage{graphicx}
\usepackage{titling}
\usepackage{listings}
\lstset{%
  basicstyle=\scriptsize\sffamily,%
  commentstyle=\footnotesize\ttfamily,%
  frameround=trBL,
  frame=single,
  breaklines=true,
  showstringspaces=false,
  numbers=left,
  numberstyle=\tiny,
  numbersep=10pt,
  keywordstyle=\bf
}
\newcommand{\subtitle}[1]{%
  \posttitle{%
    \par\end{center}
    \begin{center}\large#1\end{center}
    \vskip0.5em}%
}

%%%%%%%%%%%%%%%% PAGE DE GARDE %%%%%%%%%%%%%%%%%%%%%%
% Crédit : http://www.grappa.univ-lille3.fr/FAQ-LaTeX/6.67.html
\newlength{\larg}
\setlength{\larg}{14.5cm}

%\title{
%{\rule{\larg}{1mm}}\vspace{7mm}
%\begin{tabular}{p{2cm} r}
%   & {\Huge {\bf Méthodes numériques de base}} \\
%   & \\
%   & {\huge Cours de première année - ENSIMAG}
%\end{tabular}\\
%\vspace{2mm}
%{\rule{\larg}{1mm}}
%\vspace{2mm} \\
%\begin{tabular}{p{11cm} r}
%   & {\large \bf } \\
%   & {\large }
%\end{tabular}\\
%\vspace{5.5cm}
%}
%\author{\begin{tabular}{p{13.7cm}}
%    \begin{tabular}{ll}
%        Cours : & Hahmann S.\\
%         & James G.\\
%    \LaTeX : & Poupin P.
%    \end{tabular}
%\end{tabular}\\
%\hline }
\title{OS}
\author{Guillaume Huard \\
	\LaTeX : Poupin Pierre Rouby Thomas
}
\date{}

\begin{document}
\maketitle
\tableofcontents
\documentclass[a4paper,10pt]{article}

\usepackage{graphicx}
\usepackage{titling}
\usepackage{listings}
\lstset{%
  basicstyle=\scriptsize\sffamily,%
  commentstyle=\footnotesize\ttfamily,%
  frameround=trBL,
  frame=single,
  breaklines=true,
  showstringspaces=false,
  numbers=left,
  numberstyle=\tiny,
  numbersep=10pt,
  keywordstyle=\bf
}
\newcommand{\subtitle}[1]{%
  \posttitle{%
    \par\end{center}
    \begin{center}\large#1\end{center}
    \vskip0.5em}%
}
\title{Operating Systems}
\subtitle{Threads}
\author{}
\date{04/11/2014}

\begin{document}
\maketitle
%\begin{abstract}
%This document is our report of the first practical session. It contains our design choices along with the results of our implementation.	
%\end{abstract}


\section{Introduction}

A thread is an execution context that belong to a process.
A process might contain several threads which share some resources : memory and file descriptors.

Sometimes called lightweight process.
Process with single thread : code, data, and files in common, single registers and stack. 
\begin{center}
\begin{tabular}{|ccc|}
\hline
	Code & Data & Files\\
	\hline
	Register & & Stack\\
	\hline
\end{tabular}

\end{center}
Process with multiple threads : code, data, and files in common, multiple registers and stacks.

\begin{center}
\begin{tabular}{|ccc|}
\hline
	Code & Data & Files\\
	\hline
	Register & & Stack\\
	\hline
\end{tabular}

\end{center}
Process with single thread : code, data, and files in common, single registers and stack.
Process with multiple threads : code, data, and files in common, multiple registers and stacks.

Advantages :
lighter management (especially context switch)
take advantage of concurrency within a process ( eg can perform a computation during a blocking system call in another thread).
Communication between threads is easier/more efficient than IPC (Inter Process Communication) between processes.

Example : Webserver
The main thread can listen to connextions while other threads handle requests
Accesses to Webserver data can be performed concurrently and overlapped with computations

\section{Thread Models}

Threads might be :\\

Preemptive : threads might be interrupted asynchronously to switch to another thread. \\
Cooperative : the thread itself release the PCU to let another thread be scheduled.\\

Advantages/Drawbacks :\\

For preemptive : \\
Insensitive to misbehaving threads.\\
Can take advantage of multiple CPUS.\\
higher cost (context switch) \\

For cooperative :\\
easier to programm and debug  because not sensitive to data races ( will be explained later) \\
can only take advantage of a single CPU \\
more efficient


\section{Implementation}


 \end{document}

\documentclass[a4paper,10pt]{article}

%%%% PRATIQUE POUR LES ALINEAS CHIANTS
\usepackage{indentfirst}

%%%% POUR L'OPTION LABEL= %%%
\usepackage{enumitem}

\setlength{\parindent}{30pt}
\setlength{\parskip}{1ex}
\setlength{\textwidth}{15cm}
\setlength{\textheight}{24cm}
\setlength{\oddsidemargin}{0.2cm}
\setlength{\evensidemargin}{-.7cm}
\setlength{\topmargin}{-.5in}

\usepackage{graphicx}
\usepackage{titling}
\usepackage{listings}
\lstset{%
  basicstyle=\scriptsize\sffamily,%
  commentstyle=\footnotesize\ttfamily,%
  frameround=trBL,
  frame=single,
  breaklines=true,
  showstringspaces=false,
  numbers=left,
  numberstyle=\tiny,
  numbersep=10pt,
  keywordstyle=\bf
}
\newcommand{\subtitle}[1]{%
  \posttitle{%
    \par\end{center}
    \begin{center}\large#1\end{center}
    \vskip0.5em}%
}
\title{\textbf{Synchronization}}
\subtitle{M1 MoSIG : Operating Systems}
\author{Poupin Pierre \and Rouby Thomas}
\date{04/11/2014}

\begin{document}
\maketitle
%\begin{abstract}
%This document is our report of the first practical session. It contains our design choices along with the results of our implementation.	
%\end{abstract}



\section{Semaphores}

Proposed by Dijkstra, Semaphores are special counters on which 2 operations are defined :
\begin{itemize}
  \item P or wait on a semaphore s :
    \begin{verbatim}
      if (s.counter == 0)
        wait
        s.counter--
    \end{verbatim}
  \item V or post on a semaphore s :
  \begin{verbatim}
      s.counter++
  \end{verbatim}
\end{itemize}

\subsection{Typical use :} 

\begin{itemize}
  
\item restrict the access to a region to a fixed number of threads.
Example :

\begin{verbatim}
    s = semaphore initialized to N
    threads execute :
        wait(s);
            restricted section;
        post(s);
\end{verbatim}

\item as a lock (binary semaphore).
Example :
\begin{verbatim}
  s = semaphore initialized to 1
  Some code as above for threads.
\end{verbatim}
\end{itemize}

\subsection{Example: Procuder/Consumer}

\begin{figure}
  \begin{center}
    \includegraphics[scale=0.5]{shared_queue.png}
    \caption{Situation of the producer/consumer example}
    \label{}
  \end{center}
\end{figure}

In this example, we will use a circular buffer of size N as our queue.
If an operation is not possible, threads wait until the operation is possible.
\begin{verbatim}
semaphore s = 1, available = N, occupied = 0;

produce(int object) {

     //possibility to wait
     wait(available);
     // at most N producers
     wait(s);
     queue.buffer[(queue.head + queue.size) % N ] = object;
     queue.size++;
     post(s);
     post(occupied);
}

int consume() {
     int result;
     wait(occupied);
     // number of consumers waiting bounded...
     wait(s);
     result = queue.buffer[queue.head];
     queue.head = (queue.head +1) % N;
     queue.size--;
     post(available);
     post(s);
}
//Idea : use semaphores to count :
// the number of available slots (semaphore initialized to N)
// the number of occupied slots (semaphore initialized to 0)

Comment on the code: complicated because the order of wait operation matters.
\end{verbatim}



\subsection{Implementation of Semaphores :}

A semaphore is :
\begin{itemize}
  \item an integer value : counter
  \item a list of blocked processes : l
  \item a boolean variable : lock
\end{itemize}
\begin{verbatim}
  typedef struct{
  int lock;
  list l;
  int counter;
  } semaphore_t;
  
  // initially, lock=0, l is empty, counter is defined at semaphore creation
  
  void wait(semaphore_t s){
       while(test_and_set(&s.lock,1) {}
       if (s.counter>0){
            s.counter--;
            s.lock =0;
       }
       else{
            tid = get_current_thread_id();
            unschedule(tid);
            mark_as_blocked(tid);
            insert(s.l,tid);
            s.lock = 0;
            schedule();
       }
  }
  
  void post(semaphore_t s) {
       while(test_and_set(&s.lock,1) {}
       if (is_empty(s.l)){
            s.counter++;
            s.lock =0;   
       }
       else {
            tid = remove(s.l);
            mark_as_unblocked(tid);
            s.lock = 0;
       }
  }
  
  Comment : the wait operation takes advantage of the scheduler to perform its task.
\end{verbatim}

\section{Monitors}

Introduced by Haare, monitors are objects in which methods are executed in mutual exclusion.

\begin{tabular}{cccc}
  threads & \vdots & \vdots & \vdots
\end{tabular}
|
| 
\textgreater
\begin{tabular}{|c|}
 attributes \\
 : \\
 : \\
 : \\
\end{tabular}

Only one thread at a time can execute something inside the monitor .

This model makes things simpler because it is not necessary to manage locks anymore.

\subsection{Example: Producer/Consumer}

The queue will be implemented as a monitor, with buffer, size, head as attributes and two methods :

Thr monitors support two operations :
\begin{itemize}
  \item wait(condition C) : releases the access to the monitor then waits until some signal is sent on C, then compete for the lock to retrieve access to the monitor.
  \item signal(condition C) : sends a signal on C. The signal is lost if no thread waits on C.
\end{itemize}
 
 
 \begin{verbatim}
 
  condition not_full , not_empty;
  
   produce(int object){
        while(size==N)
        wait(not_full);
        buffer[(head+size)%N] =object;
        size++;
        signal(not_empty);
   }
   
   int consume(){
        int result;
        while(size==0)
        wait(not_empty);
        result = buffer[head];
        head =(head+1)%N;
        size--;
        signal(not_full);
   }
 \end{verbatim}


\subsection{Implementation}

Similar to semaphores. Notice that both models are equivalent. It is possible to implement one of them using the other.
\end{document}

\documentclass[a4paper,10pt]{article}

%%%% PRATIQUE POUR LES ALINEAS CHIANTS
\usepackage{indentfirst}

%%%% POUR L'OPTION LABEL= %%%
\usepackage{enumitem}

\setlength{\parindent}{30pt}
\setlength{\parskip}{1ex}
\setlength{\textwidth}{15cm}
\setlength{\textheight}{24cm}
\setlength{\oddsidemargin}{0.2cm}
\setlength{\evensidemargin}{-.7cm}
\setlength{\topmargin}{-.5in}

\usepackage{graphicx}
\usepackage{titling}
\usepackage{listings}
\lstset{%
  basicstyle=\scriptsize\sffamily,%
  commentstyle=\footnotesize\ttfamily,%
  frameround=trBL,
  frame=single,
  breaklines=true,
  showstringspaces=false,
  numbers=left,
  numberstyle=\tiny,
  numbersep=10pt,
  keywordstyle=\bf
}
\newcommand{\subtitle}[1]{%
  \posttitle{%
    \par\end{center}
    \begin{center}\large#1\end{center}
    \vskip0.5em}%
}
\title{\textbf{Deadlocks}}
\subtitle{M1 MoSIG : Operating Systems}
\author{Rouby Thomas}
\date{18/11/2014}

\begin{document}
\maketitle
%\begin{abstract}
%This document is our report of the first practical session. It contains our design choices along with the results of our implementation.	
%\end{abstract}

\section{Example}

  \begin{center}
  s = 1;
  available = N;
  occupied = 0;
  
    \begin{tabular}{cc}
     \textbf{T1} & \textbf{T2} \\
      wait(s) & wait(s)\\
     (*) wait(available) & wait(occupied) \\
      . & . \\
      . & . \\
      . & . \\
      post(occupied) & post(available) \\
      post(s) & post(s) \\
    \end{tabular}
  \end{center}

T1 holds s and wait for available. Available will be release by T2 which waits for s.
If the thread T1 waits in (*) instruction, there will be a deadlock.

In general, a deadlock is a situation in which each thread in a set of threads hold a resource and is waiting for a resource held by another thread.
It is only possible if :
\begin{itemize}
  \item mutual exclusion : a resource held by a thread cannot be acquired by another.
  \item no premption : a resource cannot be taken, even temporarily, from a thread which holds it.
  \item hold and wait : all the threads in the set hold at least one resource and wait for another one.
  \item circular wait : t1 waits for t2 to release its resources. t2 waits for t3, t3 waits for t4... tn waits for t1 .
\end{itemize}

To avoid deadlocks, there are several strategies :
\begin{itemize}
  \item manage to avoid one of the conditions :
  \begin{itemize}
    \item share resources so that all threads can access concurrently to them. But it is not always possible.
    \item preemption : possible with CPU, but still not always possible.
    \item hold and wait : we could require from threads to acquire at once all the resources they need. But it is not practical, not efficient.
    \item circular wait : enforce an order for acquiring resources. But it is too difficult in dynamic setups.
  \end{itemize}
  \item detect deadlocks.
  \item prevent deadlocks.
\end{itemize}

\section{Resource allocation graph}

We need some internal representation of the current state of the system. We are in a situation in which there are mutual exclusion, no preemption and hold and wait $=>$  We need to know if circular wait is present/possible.

Model : resources \& threads are vertices of a graph.
\begin{figure}[h]
  \begin{center}
    \includegraphics[scale =0.6]{resource_allocation_graph.png}
    \caption{Resource allocation graph}
  \end{center}
\end{figure}

There is a circular wait in the system if there is a cycle in this graph.
$=>$ Detecting a deadlock is easy using this graph ( using, for instance, a bellman-ford algorithm).

Another way to use this graph is to perform a cycle detection for each new request and to decide to refuse it if this induces a cycle in the graph.
This model is not suited if resources are divided into types and if threads ask for one instance of a given resource.
It can be extended :

\begin{figure}[h]
  \begin{center}
    \includegraphics[scale=0.6]{resource_graph_extended}
    \caption{Extended resource allocation graph}
  \end{center}
\end{figure}

In this model, the presence of a cycle doesn't necessarily means that there is a deadlock, only a set of cycles that saturates all the resources it contains, will be a deadlock. $-->$ Not practical.

\section{Allocation matrices and banker algorithm}

\subsection{Hypothesis}

At first we assume that we know :

\begin{itemize}
  \item a vector Available such that : Available[i] = number of instances of resources i that are available.
  \item a matrix Allocation such that : Allocation[i][j] = number of resources of type i allocated to thread j.
  \item a matrix Max such that : Max[i][j] = maximal number of resources of type i the thread j will need.
  \item a matrix Need such that : Nedd[i][j] = Max[i][j] - Allocation[i][j]
\end{itemize}

We will try to find if there is an order that make the execution of all threads possible :

\begin{enumerate}
  \item Work = Available
  \item Finished[j] = false for all threads.
  \item while there is some thread j such that : \\ Finished[j] = false \&\& Noeud[i][j] $\leq$ Work[i] $\forall$, \\then : Work[i] += Allocation[i][j] $\forall$ i ; \\Finished[j] =true;
  \item if there is a j such that Finished[j] = false, the system is not in a safe state.
\end{enumerate}

This previous algorithm finds out if the system is in a safe state or not.

For deadlock prevention, we use the banker algorithm : when an allocation request is issue to the system :

\begin{enumerate}

  \item Pretend the request is granted : \\ Available -= request; \\ Allocation[*][j] += request;
  \item run the safe state detection
  \item if the state is not safe, rollback and deny the request.
  
\end{enumerate}

\subsection{Example of the banker algorithm}


  \begin{center}
  Threads
    \begin{tabular}{ccccccc}
       Allocation& A B C & &Available & & Need & A B C\\
       0 & 1 0 1 & & (\textbf{0})1 0 2 & & 0 & 0 2 0 \\
       1 & 0 0 1 & &  & & 1 & 2 1 0\\
       2 & 0 2 0 & &  & & 2 & 0 0 1\\
       3 & (\textbf{2})1 0 0 & &  & & 3 & (\textbf{1})2 2 0\\
       4 & 2 0 0 & &  & & 4 & 2 2 3\\
    \end{tabular}
  \end{center}

Assume that thread 3 requests 1 A resource (1,0,0) (represented in the example with the new numbers between paranthesis)
\begin{enumerate}
  \item Pretend the request is accepted
  \item Run the algorithm to know if the state is safe : Work = (0,0,2). 

  \begin{center}
    \begin{tabular}{cc}
      Thread & Finished\\
      0 & f\\
      1 & f\\
      2 & f\\
      3 & f\\
      4 & f\\
    \end{tabular}
  \end{center}

Thread 2 has needs $\leq$ Work  : Work = (0,2,2).

Thread 0 has needs $\leq$ Work : Work = (1,2,3).

Thread 3 has needs $\leq$ Work : Work = (3,2,3).

Thread 4 has needs $\leq$ Work : Work = (5,2,3).

Thread 1 has needs $\leq$ Work : Work = (5,2,4).

So now we get :

\begin{center}
    \begin{tabular}{cc}
      Thread & Finished\\
      0 & t\\
      1 & t\\
      2 & t\\
      3 & t\\
      4 & t\\
    \end{tabular}
  \end{center}

\item Finished is filled with "true" $=>$ the request can be granted.
\end{enumerate}

Assume that thread 4 requests 1A resource and 1C resource (1,0,1).

\begin{enumerate}
  \item if the request is $>$ Need(thread), or $>$ Available, deny it.
  \item Pretend the request is accepted.
  \item run the algorithm to know if the state is safe : Work =(0,0,1)
  
Thread 2 has needs $\leq$ Work  : Work = (0,2,1).

Thread 0 has needs $\leq$ Work : Work = (1,2,2).

Thread 3 has needs $\leq$ Work : Work = (4,2,3).

Thread 4 has needs $\leq$ Work : Work = (5,2,3).

Thread 1 has needs $\leq$ Work : Work = (5,2,4).

So now we get again :

\begin{center}
    \begin{tabular}{cc}
      Thread & Finished\\
      0 & t\\
      1 & t\\
      2 & t\\
      3 & t\\
      4 & t\\
    \end{tabular}
  \end{center}
  
  \item the request is granted
\end{enumerate}

Running time of the safe state algorithm is $n^2*m$ where n is the number of threads and m the number of resource types $=>$ Costly, repeated for each allocation request.


Global method : you compare the available vector with the needs vectors.
If there is any need vector which is lower than available, the threads order will start with this one, and will release his allocated resources.
Therefore, Thread order += Thread X and Available += Allocation of X
Then, you repeat until all threads are ordered or when you are stuck.
All states will be safe or one won't be.

\section{Detecting Deadlocks vs Preventing them}

Let run the system without doing anything and look for a deadlock periodically.
Using the resource allocation graph, it's the same algorithm to prevent or detect : find a cycle in the graph.
using Allocation matrix : we take an optimistic approach, we assume that thread don't need any more resources $=>$ we replace the "Need" Matrix with pending requests, then use the safe state detection to find out if there is a deadlock

If a deadlock is detected :
\begin{itemize}
  \item preempt resources $->$ not always possible.
  \item kill some of the threads :
  \begin{itemize}
    \item the minimal number of threads that restores a safe state ?

$=>$ risk of killing important threads ( ex : Window manager, command interpreter...)
    \item killing threads that are not related to the system.
    \item let the user decide.
    \item choose at random...
  \end{itemize}
\end{itemize}

In most systems, deadlocks are not handled at all.
One canf find them in :
\begin{itemize}
  \item some experimental OS
  \item some debugging environments
\end{itemize}



\end{document}

\documentclass[a4paper,10pt]{article}

%%%% PRATIQUE POUR LES ALINEAS CHIANTS
\usepackage{indentfirst}

%%%% POUR L'OPTION LABEL= %%%
\usepackage{enumitem}

\setlength{\parindent}{30pt}
\setlength{\parskip}{1ex}
\setlength{\textwidth}{15cm}
\setlength{\textheight}{24cm}
\setlength{\oddsidemargin}{0.2cm}
\setlength{\evensidemargin}{-.7cm}
\setlength{\topmargin}{-.5in}

\usepackage{graphicx}
\usepackage{titling}
\usepackage{listings}
\lstset{%
  basicstyle=\scriptsize\sffamily,%
  commentstyle=\footnotesize\ttfamily,%
  frameround=trBL,
  frame=single,
  breaklines=true,
  showstringspaces=false,
  numbers=left,
  numberstyle=\tiny,
  numbersep=10pt,
  keywordstyle=\bf
}
\newcommand{\subtitle}[1]{%
  \posttitle{%
    \par\end{center}
    \begin{center}\large#1\end{center}
    \vskip0.5em}%
}
\title{\textbf{Synchronization without locks}}
\subtitle{M1 MoSIG : Operating Systems}
\author{Poupin Pierre \and Rouby Thomas}
\date{18/11/2014}

\begin{document}
\maketitle
%\begin{abstract}
%This document is our report of the first practical session. It contains our design choices along with the results of our implementation.	
%\end{abstract}


Synchronization primitives are costly :
\begin{itemize}
  \item spinlocks :
  \begin{itemize}
    \item atomic instruction that locks the memory bus
    \item active wait
  \end{itemize}
  \item semaphore/monitor :
  \begin{itemize}
    \item risk of being unscheduled : cost of switch
    \item entering in the kernel
  \end{itemize}
\end{itemize}

Notice that actual implementation of locks in system such as Linux is a mix between spinlock and unscheduling of the thread.

Other solutions :

\begin{itemize}
  \item algorithmic changes : no more critical sections
  
  Example : Producer/Consumer
  
  \begin{center}
size\\buffer\\head

      \begin{tabular}{l|l}
        Producer & Consumer\\
		\hline
        while(size==N) {} & while(size==0) {}\\
        buffer[(head+size)\%N]= object & result = buffer[head]\\
         & head=(head+1)\%N \\        
        size++ & size -- \\
      \end{tabular}
    \end{center}
  
  
  \begin{itemize}
    \item forget about size
    \item use head and tail
    \item single producer/consumer
    \item SC
  \end{itemize}

So we get :
\begin{center}
tail\\buffer\\head

      \begin{tabular}{l|l}
        Producer & Consumer\\
		\hline
        while((tail+1)\%N==head) {} & while(head==tail) {}\\
        buffer[tail]= object & result = buffer[head]\\
        tail = (tail+1)\%N & head=(head+1)\%N \\        
      \end{tabular}
    \end{center}    

\item take advantage of atomic instructions

Example : 
\begin{verbatim}

Compare_and_Swap(address,old,new)
 if(*address ==old) {
    *address=new;
    return 1;
 }
 else return 0;
\end{verbatim}
can be used to write an atomic stack :

\begin{verbatim}
  void atomic_push(stack* s, int value) {
      item = malloc(sizeof(node));
      item->val = value;
      do {
          item->next = *s;
      } while (compare_and_swap(s, item->next, item);
  }
\end{verbatim}

Similar for atomic\_pop (try to set s to *s-$>$next atomically)

\end{itemize}

Other performance issues include :

\begin{itemize}
  \item locks locality: with non uniform memories or with non uniform caches. $=>$ more elaborate locking structures (hierarchical, or distributed and made coherent)
  \item bottlenecks: if a single lock is used by all the threads $=>$ it's better to separate the work into independant parts and use several locks.
\end{itemize}

\end{document}

\documentclass[a4paper,10pt]{article}

%%%% PRATIQUE POUR LES ALINEAS CHIANTS
\usepackage{indentfirst}

%%%% POUR L'OPTION LABEL= %%%
\usepackage{enumitem}

\setlength{\parindent}{30pt}
\setlength{\parskip}{1ex}
\setlength{\textwidth}{15cm}
\setlength{\textheight}{24cm}
\setlength{\oddsidemargin}{0.2cm}
\setlength{\evensidemargin}{-.7cm}
\setlength{\topmargin}{-.5in}

\usepackage{graphicx}
\usepackage{titling}
\usepackage{listings}
\lstset{%
  basicstyle=\scriptsize\sffamily,%
  commentstyle=\footnotesize\ttfamily,%
  frameround=trBL,
  frame=single,
  breaklines=true,
  showstringspaces=false,
  numbers=left,
  numberstyle=\tiny,
  numbersep=10pt,
  keywordstyle=\bf
}
\newcommand{\subtitle}[1]{%
  \posttitle{%
    \par\end{center}
    \begin{center}\large#1\end{center}
    \vskip0.5em}%
}
\title{\textbf{Low level inputs/outputs}}
\subtitle{M1 MoSIG : Operating Systems}
\author{Poupin Pierre \and Rouby Thomas}
\date{18/11/2014}

\begin{document}
\maketitle

\section{Introduction}

Architecture of a typical device :

\begin{figure}[h!]
  \begin{center}
    \includegraphics[width=0.8\textwidth]{architecture_device.png}
    \label{Architecture of a typical device}
  \end{center}
\end{figure}
This general scheme applies to : hard drives, printers, mouses...


Computer layout :
\begin{figure}[h!]
  \begin{center}
    \includegraphics[width=0.8\textwidth]{computer_layout.png}
    \label{fig:}
  \end{center}
\end{figure}

To access to periphericals, one has to issue either :

\begin{itemize}
  \item special instructions to write to one of the I/O registers of periphericals.
  A port is provided which is the target device.
  
  \item memory access ... . The address space is divided into :
  \begin{itemize}
    \item  main memory
    \item 16 registers of periphericals
  \end{itemize}
\end{itemize}

Basic exchange with a peripherical :

\begin{verbatim}
  while(read(status) == busy) 
  {}
  write(data,some_data);
  write(control,some_command);
  while(read(status)==busy) 
  {}
\end{verbatim}



\section{Device Driver}

The OS would rather handle generic periphericals: for instance, a storage device should look like a sequence of data block

%place table here

Along with two properties :
\begin{itemize}
  \item 
\end{itemize}


\end{document}

\chapter{Input/Output scheduling}
\section{Introduction, hard drives technology}

Algorithms for I/Os scheduling in use today are motivated by mechanical hard drives technology.
A mechanical hard drive is made :
\begin{itemize}
  \item platters (rotating) on which data is stored.
  \item arm to move r/w heads.
  \item r/w head at the end of the arm.
\end{itemize}

\begin{figure}[h!]
  \begin{center}
    \includegraphics{hard_drive_picture.jpg}
  \end{center}
\end{figure}

%place hard drive picture here
To perform an access to a specific sector :
\begin{itemize}
  \item move the arm to the proper track
  $=>$ takes time: depends on the engine moving the arm.
  \item wait for the desired sector
   $=>$ takes time: depends on the rotating speed of the platters.
  \item select the head and read or write
  $=>$ fast.
\end{itemize}

As a result, the access time to some sector will be :
\begin{itemize}
  \item fast: if the arm is already on the right track and the sector already under the r/w head. (about 125 MB/s $=>$ 500 nanoseconds to get 4 bytes).
  It is the bandwidth of sequential accesses.
  \item slow: otherwise (about 10 ms to get 4 bytes : there is a 20 000 factor).
  It is the latency of a random access.
  
\end{itemize}

Usually, sequential accesses can reach full bandwidth with a small degradation on track change.
Latency is given on average, the exact latency depends on the istance of the arm from the track and on the sector position.

\section{I/Os scheduling}

The CPU is much faster than a hard disk, even at full bandwidth, there is a possibility to produce more I/O requests than the drive can handle

$=>$ queue of pending requests in the kernel

$=>$ optimizations: reordering, aggregation.

\subsection{FcFs (First come First served)}

\begin{figure}[h!]
  \begin{center}
    \includegraphics[width=0.6\textwidth]{fcfs.png}
  \end{center}
\end{figure}

\subsection{SSTF ( SSF, NBF)}

Shortest seek-time first of shorter seek first or nearest block first.

\begin{figure}[th!]
  \begin{center}
    \includegraphics[width=0.6\textwidth]{sstf.png}
  \end{center}
\end{figure}

But :

\begin{itemize}
  \item adjacent accesses are the best if they are on the same track
  
  $=>$ should be improved by taking into account :
  
  
  (The geometry of the drive is not given to the OS). 
  \begin{itemize}
    \item number of tracks to cross
    \item time to wait for the sector once on the right track
  \end{itemize}
 
  
  $=>$ might be implemented in the disk itself (SCSI and SATA)
  
  \item starvation: if requests close the current head position arrive constantly.
  This is especially true for requests close to the ends of the disk.
\end{itemize}

\subsection{SCAN/LOOK/the elevator}

This algorithm serve requests in two phases :

\begin{itemize}
  \item first by increasing order of sector number
  \item secondly by decreasing order of sector number
\end{itemize}
Then it loops over these two phases.

\begin{figure}[h!]
  \begin{center}
    \includegraphics[width=0.8\textwidth]{elevator.png}
    \label{fig:1}
  \end{center}
\end{figure}

Advantages :

\begin{itemize}
  \item Some locality, close requests already in the queue will be grouped
  \item no starvation, the head moves strictly toward the last request in one direction.
\end{itemize}

Drawbacks :

\begin{itemize}
  \item accesses in the wrong direction are not necessarily efficient in the disk.
  \item requests on the middle of the disk have a better serving time $=>$ fairness issue
\end{itemize}

\subsection{Circular SCAN}

Serve request only by increasing sector number. Rewind to the request nearest to the beginning of the disk when the request with the largest sector number is served.
\begin{figure}[h!]
  \begin{center}
    \includegraphics[width=0.8\textwidth]{circular.png}
    \caption{Circular Scan}
    \label{fig:2}
  \end{center}
\end{figure}
\subsection{Anticipatory scheduling}

Do not apply the work-first principle; when a process is performing a sequential access, all the requests are not necessarily in the queue.

The idea is to wait a little time after a request to let the opportunity to the served process to issue the next request

\subsection{Other strategies}
\begin{itemize}
  \item Deadline scheduling :
  requests are served by optimizing locality (SSTF with anticipatory waits) but are also associated with a deadline after which they will be served in priority.
  
  $=>$ no starvation
  
  \item fair queuing :
  each process has its own queue, the scheduler divied "fairly" the time during which a process can have its requests served.
\end{itemize}

\section{Conclusion}

These algorithms assume that :
\begin{itemize}
  \item sequential accesses are much faster than random ones (because of the 20 000 factor)
  \item it is worth spending time reordering requests ( very large gain).
  \item seek time depends on the distance between sectors.
\end{itemize}

But it is not completely true with newer technologies (SSD).



\chapter{Disk Technologies and their impact on OS}

\section{RAID}

Redundant Array of Inexpensive Disks is made of several ordinary disks assembled into one logical larger disk. The objective might be to improve :
\begin{itemize}
  \item performances
  \item reliability
  \item both
\end{itemize}

\subsection{RAID 0}

Also called striping, aims at improving performances.
\begin{figure}[h!]
  \begin{center}
    \includegraphics[width=0.6\textwidth]{raid_0.png}
  \end{center}
\end{figure}

A logical disk in which block n is the block $(n/\#disks)$ of disk $(n \% \#disks)$

Actually the logical disk is divided into chunks which might be larger than the physical block

Also called striping, aims at improving performances.
\begin{figure}[h!]
  \begin{center}
    \includegraphics[width=0.8\textwidth]{raid_0_chunks.png}
  \end{center}
\end{figure}

Advantages:

\begin{itemize}
  \item for large sequential accesses, requests can be issued in parallel on all the disks
  \item some random accesses might be issued in parallel, if they are related to chunks on different disks.
  \item $=>$ Increased Bandwith.
\end{itemize}

Drawbacks:

\begin{itemize}
  \item latency is not improved and might even be degraded
  \item more prone to failure: the probability of failure of the array is larger than the probability of failure of one of the disk
\end{itemize}

\subsection{RAID 1}

Also called mirroring, aims at improving reliability and might improve performances.

\begin{figure}[h!]
  \begin{center}
    \includegraphics[width=0.7\textwidth]{raid_1.png}
  \end{center}
\end{figure}

\begin{itemize}
  \item Read accesses can be issued to any of the disks
  \item write accesses are broadcasted in parallel to all the disks
\end{itemize}

Advantages:

\begin{itemize}
  \item performance is improved if read accesses are dispatched on several disks (not always implemented)
  \item can tolerate a number of disks failures of (n-1) (where n is the number of disks)
\end{itemize}

Drawbacks:

\begin{itemize}
  \item n*disks capacity results in a logical disk which has the capacity of a single disk. 
\end{itemize}

\subsection{Hybrid Setups : 0+1, 1+0}

\begin{figure}[h!]
\centering
        \begin{subfigure}[b]{0.3\textwidth}
                \includegraphics[width=\textwidth]{0_plus_1.png}
                \caption{0+1 setup}
        \end{subfigure}
        \begin{subfigure}[b]{0.3\textwidth}
                \includegraphics[width=\textwidth]{1_plus_0.png}
                \caption{1+0 setup}
        \end{subfigure}
\end{figure}

Characteristics of the 0+1 setup:
\begin{itemize}
  \item More efficient than a single disk
  \item tolerate 1 failure in the worst case
  \item loss of capacity depends on the number of mirrored disks
\end{itemize}
Characteristics of the 1+0 setup:
Same as the previous one. 

\subsection{RAID 4}
Combines performance and reliability without loosing too much capacity.
\vspace{0.5cm}
\begin{figure}[h!]
  \begin{center}
    \includegraphics[width=0.8\textwidth]{raid_4.png}
  \end{center}
\end{figure}

pO contains the parity information computed from chunks 0,1 and 2.

pX contains the parity information computed from chunks X*n, X*n+1, ... , X*n + n-1.

\begin{figure}[h!]
  \begin{center}
    \includegraphics[width=0.5\textwidth]{parity_disk.png}
  \end{center}
\end{figure}

Advantages :

\begin{itemize}
  \item improved bandwith, similar to a RAID 0 setup with n-1 disks.
  \item tolerates the failure of one disk :
  \begin{itemize}
    \item if this is the parity disk :
    \begin{itemize}
      \item $=>$ no data loss
      \item $=>$ performance remains the same
    \end{itemize}
    \item otherwise :
    \begin{itemize}
      \item lost data can be recomputed : read to all the disk and XOR the results.
      \item large loss of performance.
    \end{itemize}
  \end{itemize}
  \item in the case of a failure, the failing disk has to be replaced and the array rebuild (long operation)
\end{itemize}

Drawbacks :
Small writes : two independant writes to two chunks on different disks cannot be performed in parallel because of the parity disk.

\subsection{RAID 5}

Like RAID 4, except that the parity chunk is stored in a cyclic manner on all the disks.
\begin{center}
\begin{table}[h]
\begin{tabular}{|l|l|l|l|l|l|l|}
\cline{1-1} \cline{3-3} \cline{5-5} \cline{7-7}
$P_0$ &  & 0  &  & 1  &  & 2  \\ \cline{1-1} \cline{3-3} \cline{5-5} \cline{7-7} 
3  &  & $P_1$ &  & 4  &  & 5  \\ \cline{1-1} \cline{3-3} \cline{5-5} \cline{7-7} 
6  &  & 7  &  & $P_2$ &  & 8  \\ \cline{1-1} \cline{3-3} \cline{5-5} \cline{7-7} 
9  &  & 10 &  & 11 &  & $P_3$
\end{tabular}
\end{table}
\end{center}

The parity information for stripe X is stored on disk $(X/n)\%n$.
So for chunk C, it is stored on disk $(X/n(n-1))\%n$.

\begin{figure}[h!]
  \begin{center}
    \includegraphics[width=0.5\textwidth]{parity_disk_2.png}
  \end{center}
\end{figure}
\vspace{0.5cm}

Advantages :
\begin{itemize}
  \item identical to the ones of RAID 4.
  \item the small writes issue is lessened because some independant write can be issued in parallel.
\end{itemize}

Exemple:

\begin{itemize}
  \item write
      \begin{tabular}{|cccc|}
      \hline
         1&1&0&0 \\
         1&0&0&1 \\
      \end{tabular}
to chunk 1. new parity = old data XOR data XOR old parity.

\item write
      \begin{tabular}{|cccc|}
      \hline
         0&0&0&1 \\
         1&1&1&0 \\
      \hline
      \end{tabular}
to chunk 5. 
\end{itemize}

\subsection{RAID 6}

Additional redundant information to tolerate the failure of two disks (computed from a polynomial in Gallois' Field).

Drawbacks: Computationally intensive, has to be hardware accelerated.

\subsection{Relationship with the OS}

There is a software implementation of levels 0 to 5 in Linux.
Built on top of the device mapper.

\begin{figure}[h!]
  \begin{center}
    \includegraphics[width=0.6\textwidth]{device_mapper.png}
  \end{center}
\end{figure}

As efficient as hardware solution on a regular desktop machine.
Other implementations are:

\begin{itemize}
  \item fully in hardware (dedicated cards)
  \item partially in hardware (eg parity information computed by the processor with intel chipsets)
\end{itemize}



\chapter{Filesystems}

\section{Introduction}

The filesystem bridges the gap between :
\begin{figure}[h!]
  \begin{center}
    \includegraphics[width=0.6\textwidth]{filesystem.png}
  \end{center}
\end{figure}

\section{User Interface}

Is made of :

\begin{itemize}
  \item files: elementary pieces of information
  \item directories: places which store files and/or directories.
\end{itemize}
  
  One particular directory: root
  \begin{figure}[h!]
  \begin{center}
    \includegraphics[width=0.5\textwidth]{root.png}
  \end{center}
\end{figure}
  
  Notice : the interface is made in a way that prevents cycles (the FS is a tree).
  
  The user can access to the FS using system calls:

    \begin{center}
      \begin{tabular}{c|c}
         open & opendir \\
         read & readir \\
         write & mkdir \\
         close & closedir \\
         unlink & \\
      \end{tabular}
    \end{center}
This interface is fixe for all existing FS in UNIX.

\section{Organization on the hard disk}

Objectives:

\begin{itemize}
  \item contiguous allocation is better
  \item avoid fragmentation
\end{itemize}

Fact:

\begin{itemize}
  \item accesses to hard disks are performed by blocks, to make worthwhile the fact of paying the latency.
  \item the filesystem can even use larger blocks because accesses (to a given file) are mostly sequential.
  
\end{itemize}

$=>$ there is a risk of increasing internal fragmentation, the block size is a compromise.

\subsection{Linking ---$>$ FAT}

Linked list of blocks, we use a part of the block to link to the next one.

 \begin{figure}[h!]
  \begin{center}
    \includegraphics[width=0.8\textwidth]{fat.jpg}
  \end{center}
\end{figure}

Directories are special files which contain a sequence of couples : filename \& number of first block (of the file).

Files and Directories start with a meta data block which contains:
\begin{itemize}
  \item type of file
  \item permissions
  \item size
  \item access times
  \item ...
\end{itemize}

Not efficient: random accesses to files are not efficient because they require the reading of all the blocks.

Idea to solve the issue, place the linking information elsewhere.

Normal use:
\begin{itemize}
  \item opening: not efficient (one has to read all the directories on the path) but performed once. $--->$ number block of metadata of the file.
  \item read: ok, jump from block to block.
  \item lseek, open(append mode) :   not efficient.
\end{itemize}

$=>$ File Allocation Table (FAT)

Still, there are several accesses when randomly accessing to a large file.

$=>$ usually the whole table is cached into memory.

\begin{figure}[h!]
  \begin{center}
    \includegraphics[width=0.8\textwidth]{fat_2.jpg}
  \end{center}
\end{figure}

\subsection{Inodes and allocation tables}

The inode combines:
\begin{itemize}
  \item meta data related to a file
  \item index of data blocks of the file
\end{itemize}

Stored on a single block, the size of the index is limited, ex: 12 entries in inodes in ext2.

\begin{figure}[h!]
  \begin{center}
    \includegraphics[width=0.5\textwidth]{inode.jpg}
  \end{center}
\end{figure}

$--->$ An inode also contain indirect links:
\begin{itemize}
  \item one single indirection link: points to a block which contains pointers to data blocks.

  \item one double indirection link: points to a block which contains pointers to data blocks which contains pointers to data blocks.
  \item one triple indirection link : points to ....
  
\end{itemize}

The assesses:
\begin{itemize}
  \item are very efficient for small files (direct links in the inode)
  \item are less efficient for large files (up to three indirect links) but it is bounded by a constant factor.
\end{itemize}

Overall organization:

 \begin{figure}[h!]
  \begin{center}
    \includegraphics[width=0.8\textwidth]{inode_2.jpg}
  \end{center}
\end{figure}

\section{Fragmentation}

Different from fragmentation in the memory allocator: here it is related to the fact that data blocks of a file are not necessarily contiguous.

 \begin{figure}[h!]
  \begin{center}
    \includegraphics[width=0.8\textwidth]{fragmentation.jpg}
  \end{center}
\end{figure}

$=>$ defragmentation tools: exchange blocks all accross the disk.

$=>$ not efficient

Another idea is to avoid fragmentation as much as possible within the FS.

Ideas: (Fast File System)
\begin{itemize}
  \item  try to keep small files close to each other
  \item try to divide the whole storage space into several local groups.
  \item spread large files into different groups by dividing them into large chunks.
\end{itemize}

Thus a compromise:
\begin{itemize}
  \item large chunks to limit seeks
  \item file divided into chunks to keep flexibility and avoid eventual fragmentation
\end{itemize}
\begin{figure}[h!]
  \begin{center}
    \includegraphics[width=\textwidth]{fragmentation_2.jpg}
  \end{center}
\end{figure}
These optimization are mainly common sense but they work very well in practice.

\section{Filesystem updates \& failures}

Ex: to append a block to a file (ex: /var/log/syslog)
\vspace{1cm}
\begin{figure}[h!]
  \begin{center}
    \includegraphics[width=0.8\textwidth]{fs_updates_failures.png}
  \end{center}
\end{figure}


An access is costly because of FS structure updates

$=>$ the FS cache accesses to memory:
\begin{itemize}
  \item as many as possible for reads
  \item during some time for writes (5-30s)
\end{itemize}
compromise between efficiency and risk of loosing data in the case of power failure.

If there is a power loss, cached writes might be:

\begin{itemize}
  \item completely in memory
  \item poartially onto disk:
    \begin{itemize}
      \item corrupted FS
      \item incorrect data blocks
    \end{itemize}
\end{itemize}


\end{document}
