\documentclass[a4paper,10pt]{article}

\usepackage{graphicx}
\usepackage{titling}
\usepackage{listings}
\lstset{%
  basicstyle=\scriptsize\sffamily,%
  commentstyle=\footnotesize\ttfamily,%
  frameround=trBL,
  frame=single,
  breaklines=true,
  showstringspaces=false,
  numbers=left,
  numberstyle=\tiny,
  numbersep=10pt,
  keywordstyle=\bf
}
\newcommand{\subtitle}[1]{%
  \posttitle{%
    \par\end{center}
    \begin{center}\large#1\end{center}
    \vskip0.5em}%
}
\title{Operating Systems}
\subtitle{Threads}
\author{}
\date{04/11/2014}

\begin{document}
\maketitle
%\begin{abstract}
%This document is our report of the first practical session. It contains our design choices along with the results of our implementation.	
%\end{abstract}


\section{Introduction}

A thread is an execution context that belong to a process.
A process might contain several threads which share some resources : memory and file descriptors.

Sometimes called lightweight process.
Process with single thread : code, data, and files in common, single registers and stack. 
\begin{center}
\begin{tabular}{|ccc|}
\hline
	Code & Data & Files\\
	\hline
	Register & & Stack\\
	\hline
\end{tabular}

\end{center}
Process with multiple threads : code, data, and files in common, multiple registers and stacks.

\begin{center}
\begin{tabular}{|ccc|}
\hline
	Code & Data & Files\\
	\hline
	Register & & Stack\\
	\hline
\end{tabular}

\end{center}
Process with single thread : code, data, and files in common, single registers and stack.
Process with multiple threads : code, data, and files in common, multiple registers and stacks.

Advantages :
lighter management (especially context switch)
take advantage of concurrency within a process ( eg can perform a computation during a blocking system call in another thread).
Communication between threads is easier/more efficient than IPC( Inter Process Communication) between processes.





 \end{document}
