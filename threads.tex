\documentclass[a4paper,10pt]{article}

%%%% PRATIQUE POUR LES ALINEAS CHIANTS
\usepackage{indentfirst}

%%%% POUR L'OPTION LABEL= %%%
\usepackage{enumitem}
\setlist[enumerate]{label=$\bullet$}

\usepackage{graphicx}
\usepackage{titling}
\usepackage{listings}
\lstset{%
  basicstyle=\scriptsize\sffamily,%
  commentstyle=\footnotesize\ttfamily,%
  frameround=trBL,
  frame=single,
  breaklines=true,
  showstringspaces=false,
  numbers=left,
  numberstyle=\tiny,
  numbersep=10pt,
  keywordstyle=\bf
}
\newcommand{\subtitle}[1]{%
  \posttitle{%
    \par\end{center}
    \begin{center}\large#1\end{center}
    \vskip0.5em}%
}
\title{Operating Systems}
\subtitle{Threads}
\author{}
\date{04/11/2014}

\begin{document}
\maketitle
%\begin{abstract}
%This document is our report of the first practical session. It contains our design choices along with the results of our implementation.	
%\end{abstract}


\section{Introduction}

A thread is an execution context that belong to a process.
A process might contain several threads which share some resources : memory and file descriptors.

Sometimes called lightweight process.

\vspace{0.2cm}
\begin{minipage}{0.4\textwidth}
    \begin{center}
        process with single thread
        \begin{tabular}{|c|c|c|}
            \hline
            code & data & files \\
            \hline
            register & & stack \\
            & \vdots & \\
            & \vdots & \\
            \hline
        \end{tabular}
    \end{center}
\end{minipage}
\begin{minipage}{0.4\textwidth}
    \begin{center}
        process with multiple threads
        \begin{tabular}{|c|c|c|}
            \hline
            code & data & files \\
            \hline
            registers & registers & registers \\
            stack& stack & stack \\
            \vdots & \vdots & \vdots \\
            \vdots & \vdots & \vdots \\
            \hline
        \end{tabular}
    \end{center}
\end{minipage}
\vspace{0.2cm}

\subsubsection*{Advantages}
\begin{enumerate}
	\item Lighter management (especially context switch)
	\item Take advantage of concurrency within a process ( eg can perform a computation during a blocking system call in another thread).
	\item Communication between threads is easier/more efficient than IPC( Inter Process Communication) between processes.
\end{enumerate}

\subsubsection*{Example : Webserver}

The main thread can listen to connections while other threads handle requests
Accesses to Webserver data can be performed concurrently and overlapped with computations

\section{Thread Models}

Threads might be :

\begin{enumerate}
	\item Preemptive : threads might be interrupted asynchronously to switch to another thread.
	\item Cooperative : the thread itself release the PCU to let another thread be scheduled.
\end{enumerate}

\subsection*{Advantages / Drawbacks}
\subsubsection*{For Preemptive}
\begin{enumerate}
	\item Insensitive to misbehaving threads.
	\item Can take advantage of multiple CPUS.
	\item higher cost (context switch)
\end{enumerate}

\subsubsection*{For Cooperative}
\begin{enumerate}
	\item easier to programm and debug  because not sensitive to data races ( will be explained later)
	\item can only take advantage of a single CPU
	\item more efficient
\end{enumerate}


\section{Implementation}


 \end{document}
